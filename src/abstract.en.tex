Abstract of Bachelor's Thesis - Academic Year 2023
\begin{center}
\begin{large}
\begin{tabular}{|p{0.97\linewidth}|}
    \hline
      \etitle \\
    \hline
\end{tabular}
\end{large}
\end{center}

~ \\
This paper proposes a new SRv6 End behavior, called End.AN.NF, integrating Linux netfilter as a network function for service function chaining by Segment Routing (SR).
End.AN.NF allows netfilter-based applications to be executed as SR-Aware applications without modification, as it applies netfilter to inner packets encapsulated in SRv6 while performing the basic SRv6 End behavior.
Furthermore, End.AN.NF utilizes the argument of the segment identifiers to mark packets.
Consequently, this enables netfilter-based applications to match the marks on packet buffers and change rules to be applied.
We implemented End.AN.NF on the Linux kernel and evaluated its performance.
The evaluation shows that End.AN.NF achieves 27\% higher throughput and 3.0 microseconds lower latency than applying netfilter to SRv6-encapsulated inner packets by End.DT4 and H.Encaps.
~ \\
Keywords : \\
\underline{1. Service Function Chaining},
\underline{2. Segment Routing}
\underline{3. SRv6}
\begin{flushright}
\edept \\
\eauthor
\end{flushright}
