卒業論文要旨 - 2023年度(令和5年度)
\begin{center}
\begin{large}
\begin{tabular}{|M{0.97\linewidth}|}
    \hline
      \title \\
    \hline
\end{tabular}
\end{large}
\end{center}

~ \\
本論文では,End.AN.NF という新しい Segment Routing over IPv6 (SRv6) End behavior を提案する.
End.AN.NF は Linux netfilter を Linux の SRv6 ルーティングインフラストラクチャへ統合する事ができる.
セグメントルーティング (SR) を利用したサービスファンクションチェイニング環境において,End.AN.NF は,
Linux netfilter を SR に対応したネットワークファンクションとして扱えるようにする.
End.AN.NF を利用する際,netfilter を利用して作成されたアプリケーションの実装を変える必要はなく,
End.AN.NF は SRv6 の基本処理である End behavior を実行しながら,SRv6 でカプセル化された内部のパケットへ netfilter を適用できる.
さらに,End.AN.NF は segment id 内の引数部分を利用して,パケットにマークを付けることができる.
したがって,netfilter を利用して作成されたアプリケーションは End.AN.NF がパケットバッファに付与したマークを照合することで,適用するルールを変更できる.
我々は End.AN.NF を Linux kernel に実装し,その性能評価を行った.
計測の結果, End.AN.NF は End.DT4 と H.Encaps を使って SRv6 でカプセル化された内部パケットに netfilter を適用する方法に比べ,
27\% 高いスループット,及び 3.0 マイクロ秒低いレイテンシを実現した.
~ \\
キーワード:\\
\underline{1. Service Function Chaining},
\underline{2. Segment Routing}
\underline{3. SRv6}
\begin{flushright}
\dept \\
\author
\end{flushright}
