\chapter{preface}
\label{chap:introduction}

% \section{本論文の目的と構成}
% 本論文における以降の構成は次の通りである.
% \ref*{chap:introduction}章では,本論文の構成について述べ,導入する.
% \ref*{chap:related_works}章では,サービスファンクションチェイニングに関する前提知識,及びそれを実現する技術について解説し,本論文の概要について述べる.
% \ref*{chap:design_and_impl}章では,本論文の提案する新たな SRv6 End behavior である End.AN.NF についての詳細な動作,及び実装について述べる.
% \ref*{chap:evaluation}章では,実装した End.AN.NF について,レイテンシ及びスループットの性能を特定のを変化させながら性能の計測する.
% \ref*{chap:conclusion}章では,本研究における結論と今後の展望について述べ,End.AN.NF に必要なネットワーク制御プレーンについて検討する.

\section{introduction}
\label{section:background}
Service Function Chaining (SFC) is a topic that has been studied in contexts of Software Defined Network (SDN) and Network Function Virtualization (NFV)~\cite{nfv,sfc-on-sdn-nfv-servey,sfc-on-sdn-scenario,imple-sfc-with-openflow}.
SFC steers packets to Network Functions (NFs) according to predetermined rules about the order and type of NFs to traverse.
% SDN controllers or routing protocols install rules, and the header of a packet may includes information about these rules.
SDN controllers or routing protocols install the rules to steer packets, and the headers of packets may include information matched by these rules.
In SFC, routers require to select the next hops for forwarding packets regardless of the shortest path.
SFC offers flexible and economical alternatives to today's static environments for Cloud Service providers (CSPs), Application Service Providers (ASPs), and Internet Service Providers (ISPs)~\cite{survey-on-sfc}.

% \sloppy There are several technology candidates that could achieve SFC, such as OpenFlow~\cite{openflow}, Network Service Header (NSH)~\cite{rfc8300}, \fussy MPLS~\cite{rfc8595} and so on
Several technology candidates could achieve SFC, such as OpenFlow~\cite{openflow}, Network Service Header (NSH)~\cite{rfc8300}, \fussy MPLS~\cite{rfc8595} and so on.
These technologies can steer packets to intended NFs based on rules, regardless of the shortest path.
In OpenFlow, a controller installs flow rules to OpenFlow switches explicitly.
OpenFlow switches forward packets to go through intended NFs with properly managed match the rules.
This architecture allows flexible path control that is not based on traditional routing protocols.
NSH identifies an NF by Service Path Identifier (SPI) and Service Index (SI).
NSH nodes forward packets according to SPI and SI by encapsulation.
NSH creates a dedicated overlay network called service plane and allows service forwarding to occur within that plane without modifying the underlying network topology. 
In MPLS, instead of using NSH directly, MPLS label stack contains the hop by hop order to pass through, routers and NFs.
This approach also achieves steering packets for SFC without modifying the underlying network topology. 

Segment Routing (SR), especially Segment Routing over IPv6 (SRv6), is also one of the technologies used for implementing SFC.
SR represents every entity, such as links, nodes, and services, in a network by \textit{segments}.
A header of a packet contains a list of the segments.
The list of segments indicates a path that the packet should pass through.
SRv6 uses an IPv6 address as an identifier for a segment, called Segment Identifier (SID).
In other words, SRv6 leverages IPv6 routing infrastructure as its underlay to deliver packets through arbitrary order of segments.
SRv6 achieves SFC by assigning segments to the NFs to be performed on those segments, and forwarding packets accordingly.

In SRv6, NFs represented by SIDs would achieve SRv6-based SFC.
However, it is not obvious how to integrate the behavior of NFs above the SRv6 layer and the underlying IPv6 routing infrastructure.
For example, let us consider a Network Address Translation (NAT) for IPv4 packets as an NF in an SRv6 network.
IPv4 packets are encapsulated in outer IPv6 headers involving SR Header (SRH). % これは IPv4 が encapsulated であることを一番先に言いたいので受動態で良さそう
% An NF receives the encapsulated packets by underlying IPv6 routing and applying IPv4 address translation for inner IPv4 packets, while it transfers the packets to satisfy a requirement of an End behavior.
If the NAT implementation is not aware of SRv6, it requires SR-proxies~\cite{ietf-spring-sr-service-programming-08} for SR-unaware NFs that introduce additional complexity~\cite{draft-scexp}.
If the implementation can perform both NAT for inner packets and the SRv6 forwarding behavior simultaneously, it would be layer violation.
Such an implementation in Linux, SERA~\cite{sera}, acts as an End behavior by modified iptables actions, although the Linux kernel has the capability of performing an End behavior in its IPv6 routing and forwarding infrastructure.

% 問題をどうやって解決したのか.この研究の提案手法の概要,評価した結果などを入れる.
% abst の中の提案手法にあたる部分を書く. 
% In this paper, we introduce an SR-Aware service function, called \texttt{End.AN.NF}, which integrates Linux netfilter as a service function while leveraging the IPv6 routing infrastructure of Linux. 
% \texttt{End.AN.NF} があると,今まで SRv6 に対応していなかった netfilter-based application が SR-Aware application になる
Along these lines, in this paper, we introduce an SR-Aware service function, called \texttt{End.AN.NF}, which enables existing netfilter-based applications to be SR-Aware applications without modification.
More specifically, it integrates Linux netfilter as an NF while leveraging the IPv6 routing infrastructure of Linux.
We designed \texttt{End.AN.NF} to treat netfilter hook points transparent to the SRv6 inner packet.
We implemented \texttt{End.AN.NF} on the Linux kernel and evaluated its throughput and latency.
The evaluation shows that \texttt{End.AN.NF} achieves 27\% higher throughput and 3.0 microseconds lower latency than applying netfilter to SRv6-encapsulated inner packets by the combination of \texttt{End.DT4} and \texttt{H.Encaps}.
In addition, the latency of \texttt{End.AN.NF} is the same as that of \texttt{End} behavior in microsecond resolution.