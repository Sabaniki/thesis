Abstract of Bachelor's Thesis - Academic Year 2023
\begin{center}
\begin{large}
\begin{tabular}{|p{0.97\linewidth}|}
    \hline
      \etitle \\
    \hline
\end{tabular}
\end{large}
\end{center}

~ \\
This paper proposes a method for integrating netfilter, a packet manipulation feature of Linux, with Segment Routing over IPv6 (SRv6), a traffic control technology.
In modern data center networks, virtualization of Network Functions (NFs) using generic servers, virtual machines, and container technologies has become widespread.
The application of NFs according to a set of rules is called Service Function Chaining (SFC).
The NFs used in SFC are called Service Functions (SFs), and SFC requires that certain packets be routed through SFs in an arbitrary order.
Traditional packet routing cannot route a packet through specific nodes in an arbitrary order.
Therefore, a routing mechanism different from traditional packet routing is required to implement SFC.
Segment Routing over IPv6 (SRv6) is one of the path control technologies that can implement SFC.
In SRv6, IP packets are encapsulated with a header called the SRv6 header, which sequentially lists the nodes that the packet will traverse.
This allows the specification of the nodes that a packet will traverse, regardless of the IP's best path, and thus the order of SF traversal can be arbitrary.
In addition to controlling traffic, SRv6 can also apply specific operations to transit packets, these specific operations being called ``behaviors''.

The Linux kernel implements a packet processing framework called netfilter, which allows packet filtering, NAT, NAPT, and other packet mangling operations.
However, netfilter cannot be applied to packets encapsulated in an SRv6 header.
Therefore, this paper introduces a new SRv6 behavior called End.AN.NF.
End.AN.NF allows Linux netfilter to be integrated into the Linux SRv6 routing infrastructure.
It allows Linux netfilter to be treated as an SRv6 compatible SF in an SFC environment using SRv6.
When using End.AN.NF, there's no need to modify the implementation of applications built with netfilter.
End.AN.NF can apply netfilter to packets encapsulated in SRv6 while performing the basic processing of SRv6, known as End behavior.
In addition, End.AN.NF can mark packet buffers, allowing applications built with netfilter to change the rules they apply by checking the mark End.AN.NF adds to the packet buffers.
This paper implements End.AN.NF in the Linux kernel and evaluates its performance.
The results show that End.AN.NF achieves 27\% higher throughput and 3.0 microseconds lower latency compared to the method of applying netfilter to packets encapsulated in SRv6 using End.DT4 and H.Encaps.


~ \\
Keywords : \\
\underline{1. Service Function Chaining}
\underline{2. Segment Routing}
\underline{3. SRv6}
\begin{flushright}
\edept \\
\eauthor
\end{flushright}