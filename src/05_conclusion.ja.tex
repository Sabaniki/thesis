\chapter{結論}
\label{chap:conclusion}
本論文では,Linux netfilter を統合し,BGP などの既存のルーティングプロトコルとの共存を実現する新しい SRv6 End ビヘイビア,End.AN.NF を提案した.
End.AN.NF は,SRv6 の内部のパケットに対して netfilter の 3 つのフックポイント prerouting,forward,postrouting を透過的に適用させることができる.
netfilter のフックポイントを透過する事により,netfilter を実装に利用して作成されたアプリケーションは,その実装を変更せずに SR-aware アプリケーションとして機能させることができる.
また,End.AN.NF はパケットをマークするために SID の \texttt{ARG} フィールドを活用する.
このアプローチにより,netfilter を内部実装に利用した SF アプリケーションは,パケットバッファ上のマークをマッチングさせることによる動的ななルール調整が可能となる.
我々は End.AN.NF を Linux カーネルに実装し,その性能を評価した.
評価の結果,我々の実装は,SRv6 インナーパケットに netfilter のルールを適用する方法である End.DT4 と H.Encaps の組み合わせと比較して,27\% 高いスループットと3.0マイクロ秒低いレイテンシを実現した.
さらに,End と End.AN.NF のスループットの差は 6\% 未満であり,End.AN.NF のオーバーヘッドは最も基本的な End の動作と比較して許容範囲内であることを示している.

