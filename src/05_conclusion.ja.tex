\chapter{結論}
\label{chap:conclusion}
本論文では,Linux netfilter を統合し,BGP などの既存のルーティングプロトコルとの共存を実現する新しい SRv6 End ビヘイビア,End.AN.NF を提案した.
End.AN.NF は,SRv6 インナーパケットに対して netfilter の 3 つのフックポイント prerouting,forward,postrouting を透過的に動作させることができる.
netfilter のフックポイントを透過する事により,netfilter ベースのアプリケーションはその実装を変更せずに SR-Aware アプリケーションとして機能させることができる.
また,End.AN.NF はパケットをマークするために SID の \texttt{ARG} フィールドを利用する.
このアプローチにより,netfilter ベースのアプリケーションはパケットバッファ上のマークをマッチングさせることによるダイナミックなルール調整が可能となる.
我々は End.AN.NF を Linux カーネルに実装し,その性能を評価した.
評価の結果,我々の実装は,SRv6 インナーパケットに netfilter のルールを適用する方法である End.DT4 と H.Encaps の組み合わせと比較して,27\% 高いスループットと3.0マイクロ秒低いレイテンシを達成した.
さらに,End と End.AN.NF のスループットの差は 6\% 未満であり,End.AN.NF のオーバーヘッドは最も基本的な End の動作と比較して許容範囲内であることを示している.
