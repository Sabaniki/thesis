卒業論文要旨 - 2023年度(令和5年度)
\begin{center}
\begin{large}
\begin{tabular}{|M{0.97\linewidth}|}
    \hline
      \title \\
    \hline
\end{tabular}
\end{large}
\end{center}

~ \\
本論文では,Linux の持つパケット操作機能である netfilter を,トラフィック制御技術の1つである SRv6 に統合する手法を提案する.
昨今のデータセンタネットワークでは,汎用的なサーバや仮想マシン,コンテナ技術を使ってネットワークの機能 (NF) を仮想化する技術が一般化してきている.
一連のルールに沿って NF を適用することを サービスファンクションチェイニング (SFC) という.
SFC にデプロイされる NF はサービスファンクション (SF) と呼ばれ,SFC ではあるパケットを任意の順番で SF へ通過させる必要がある.
従来のパケットルーティングでは,あるパケットを任意の順番で指定したノードを通過させる,ということはでききない.
よって,SFC の実現のためには従来のパケットルーティングとは別の経路制御機構が必要である.
SFC を実現できる経路制御技術の1つに Segment Routing over IPv6 (SRv6) という技術がある.
SRv6 では,SRv6 header と呼ばれるヘッダで IP パケットをカプセル化する.
また,SRv6 header にはパケットが通過するノードが順番に含まれる.
これによって,IP 的なベストパスに関係なくパケットが通過するノードを指定可能であるから,任意のルールに従って SF を通る順番を指定できる.
また,SRv6 はトラフィック制御だけでなく,トランジットするパケットに対して特定の操作を適用でき,この特定の操作の種類のことを\textbf{ビヘイビア}という.

Linux カーネルには netfilter というパケット処理フレームワークが実装されている.
netfilter を使うことで,パケットのフィルタリングや NAT,NAPT,その他のパケットマングリング操作を適用できる.
しかし,SRv6 は SRv6 header でカプセル化されているため,カプセル化されている内部のパケットに対して netfilter を適用できない.
そこで,本論文では End.AN.NF という新しい SRv6 ビヘイビア を提案する.
End.AN.NF は Linux netfilter を Linux の SRv6 ルーティングインフラストラクチャへ統合する事ができる.
End.AN.NF は,SRv6 を利用したサービスファンクションチェイニング環境において,
Linux netfilter を SRv6 に対応した NF として扱えるようにする.
End.AN.NF を利用する際,netfilter を利用して作成されたアプリケーションの実装を変える必要はなく,
End.AN.NF は SRv6 の基本処理である End ビヘイビアを実行しながら,SRv6 でカプセル化された内部のパケットへ netfilter を適用できる.
さらに,End.AN.NF は,パケットにマークを付けることができる.
したがって,netfilter を利用して作成されたアプリケーションは End.AN.NF がパケットバッファに付与したマークを照合することで,適用するルールを変更できる.
我々は End.AN.NF を Linux カーネルに実装し,その性能評価を行った.
計測の結果, End.AN.NF は End.DT4 と H.Encaps を使って SRv6 でカプセル化された内部パケットに netfilter を適用する方法に比べ,
27\% 高いスループット,及び 3.0 マイクロ秒低いレイテンシを実現した.
~ \\
キーワード:\\
\underline{1. Service Function Chaining}
\underline{2. Segment Routing}
\underline{3. SRv6}
\begin{flushright}
\dept \\
\author
\end{flushright}
