\chapter{BackGround And Related Works}
\label{chap:related_works}
SRv6~\cite{rfc8986} is one of the source routing architectures that uses IPv6.
It allows network operators or applications to specify intermediate points that a packet passes through by embedding a sequence of identifiers into the IPv6 extension header, called SRv6 header (SRH)~\cite{rfc8754}.
The identifiers are called by Segment Identifiers (SIDs), and each SID represents a specific function to be performed at a specific location in a network.  To specify which SID is the current SID in the SID list, SRH has a field called Segments Left (segleft).
segleft is an index for the SID list, starting from $(number\ of\ SIDs)-1$ and ending at zero.
A SID is an IPv6 address associated with a specific segment in a network.
The SID is structured as \texttt{LOC:FUNCT:ARG}, where \texttt{LOC} represents a locator, \texttt{FUNCT} is an identification of a local behavior associated with the SID, and \texttt{ARG} may encode additional information required for an action.
The locator may also be represented as \texttt{B:N}, where \texttt{B} is the SRv6 SID block (an IPv6 prefix allocated for SRv6 SIDs) and \texttt{N} is the identifier of the node instantiating the SID.

When SRv6 node receives a packet whose destination IPv6 address is a local SID configured in the node, the SRv6 node performs pre-defined behavior associated with the SID.
In the SRv6 context, the behaviors that SRv6 nodes perform are called End behaviors.
Currently, RFC8986~\cite{rfc8986} defines 15 types of End behaviors, and the most basic End behavior is \texttt{End}. % 自分が `pure' みたいな ワードで誤魔化していたのを 15 個あるうちの the most basic という表現にしている
\texttt{End} decrements the segleft of SRH of a received packet and replaces the destination IPv6 address with the next SID.
Next, the SRv6 node forwards the packet to the next hop in accordance with the updated destination IPv6 address.
RFC8986 also defines behaviors that encapsulate packets in SRH involving SID lists, which are called Headend behaviors.

SIDs in SRv6 are IPv6 addresses; therefore, advertising SIDs over the existing routing protocols can build a network based on SRv6.
For example, \texttt{H.Encaps} and \texttt{End.DT4}, which are Headend and End behaviors respectively, can compose layer-3 VPN~\cite{rfc9252}.
\texttt{H.Encaps} encapsulates IPv4 or IPv6 packets in outer IPv6 headers with SRH, and \texttt{End.DT4} decapsulates the packets encapsulated in the SRH.
When \texttt{H.Encaps} in an ingress SRv6 node encapsulates packets in outer IPv6 headers, whose destination address is \texttt{End.DT4} SID of an egress SRv6 node, the encapsulated packets are forwarded to the egress SRv6 node in accordance with the \texttt{LOC} of the \texttt{End.DT4} SID over the underlying SRv6 routing infrastructure.
The egress SRv6 node receives the packets and performs \texttt{End.DT4}; the packets are then decapsulated and the inner packets are routed based on a Virtual Routing and Forwarding (VRF) table associating the \texttt{ARG} of the SID.

The above examples illustrate the behaviors associated with SIDs.
On the other hand, some behaviors are not limited to encapsulation and decapsulation; NFs applied to packets can be also represented by SIDs. % パケットのヘッダを情報を書き換える,という SRv6 の基本的な動作を説明したあと,NF の適用もできる,という書き方
When an NF applies some network services for transit packets while performing \texttt{End}---decrementing segleft and updating destination IPv6 address with the next SID, such an NF is called an SR-Aware function.
SERA~\cite{sera} is an implementation of an SR-Aware function, integrated with Linux iptables.
SERA extends Linux iptables so that iptables matches fields in SRH with iptables rules, to apply filtering rules for firewall purposes.
SERA can also perform like \texttt{End}, forwarding packets toward the next SID.
This design choice makes it difficult to integrate the SIDs associated with SERA into routing infrastructures; these SIDs in iptables cannot be advertised via the existing routing protocols, unlike the example of layer-3 VPN.
As outlined above, how to integrate a form of NF and the underlying IPv6 routing infrastructure still has room for consideration. % 理由を述べた上で,検討の余地があるとする.(End.AN.NF の伏線)

In contrast to SR-Aware functions, various methodologies to integrate traditional, SR-unaware NFs into SRv6-based SFC have been proposed. % SR-aware functions を先に述べた上で unaware の特徴,動作,実例を述べる
SR Proxy~\cite{ietf-spring-sr-service-programming-08} is the key component to connect SR-unaware NFs with underlying SRv6 networks.
An SR Proxy receives packets destined to a local SID, passes inner packets without the SRH to a associated NF, attaches proper SRH to packets returned from the NF, and forwards the packets to tne next SID.
Some SR Proxy implementations in Linux have been proposed~\cite{sfc-proxy-bpf,sfc-with-leg-vnf,afxdp-for-srv6}.
On the other hand, SR Proxies fundamentally introduce additional complexities to networks.
For example, SR proxies need to determine proper SID lists attached to packets returned from NFs.
There is a possibility that inner packets have arbitrary destinations and arbitrary sources.
Thus, SID lists that the SR proxy should attach can vary depending on the inner packets.
SR proxies need to implement their own mechanism to determine proper the SID lists to be attached, for example, attaching static ones (\texttt{End.AS}) or caching some states inside proxy implementations~\cite{sfc-proxy-bpf}.
Moreover, some issues for deploying SR proxies are addressed---a type of service that cannot co-exist with a specific SR Proxy type, service liveness detection, and an SID advertisement issue for services behind SR proxies~\cite{draft-scexp}.