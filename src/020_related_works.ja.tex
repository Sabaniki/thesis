\chapter{先行研究と問題提起}
\label{chap:related_works}
本章では,Service Function Chaining (SFC),及びそれを実現する技術について解説する.
SFC を実現するための技術は複数存在する.
本論文では,複数ある技術の中で Segment Routing over IPv6 (SRv6) における SFC を前提としているため,SRv6 の概念やその知識についても解説する.

\section{SRv6 と NF の統合手法}
\label{section:srv6-and-NF}
\ref*{section:srv6}章で示した End.DT4 及び H.Encaps は,パケットをカプセル化,及びデカプセル化するビヘイビアである.
一方で,いくつかのビヘイビアのもつ機能はパケットのカプセル化,及びデカプセル化に限定されていない.
SRv6 では,パケットに適用される NF (Network Function) も SID で表現可能である.
NF がトランジットパケットに対して,segleft をデクリメントし,宛先 IPv6 アドレスを次の SID で更新する End ビヘイビアとしての動作をしながらネットワークサービスを適用する場合,その NF は SR-Aware ファンクションと呼ばれる.
SERA~\cite{sera} は,Linux iptables に統合された SR-Aware ファンクションの実装である.
SERA は Linux iptables を拡張し,SRH のフィールドと iptables のルールをマッチさせて,ファイアウォール用のフィルタリングルールを適用する.
SERA はまた,End ビヘイビアのように,パケットを次の SID に転送する機能も持つ.
しかしながら,SERA の採用した iptables を拡張する,というデザインでは SERA に関連する SID を既存のルーティングインフラに統合することは困難である.
iptables 内のフィルタリングルールとして利用するための SID の情報は,\ref*{section:srv6}章で解説した layer-3 VPN の例とは異なり,既存のルーティングプロトコルを通じて広告することはできない.
このように,NF の一形態と基盤となる IPv6 ルーティングインフラをどのように統合するかについては,検討の余地がある.

SR-Aware ファンクションと対照的に,従来の SR-unaware NF を SRv6 ベースの SFC に統合するための様々な方法論が提案されている.
SR プロキシ~\cite{ietf-spring-sr-service-programming-08} は,SR-unaware NF を SRv6 ネットワークに接続するための重要なコンポーネントである.
SR プロキシは ローカル SID 宛のパケットを受信し,SRH を持たないインナーパケットを関連する NF に渡した後,NF から返されたパケットに適切な SRH を再度付加し,次の SID にパケットを転送する.
Linux における SR プロキシの実装もいくつか提案されている~\cite{sfc-proxy-bpf,sfc-with-leg-vnf,afxdp-for-srv6}.
しかし,SR プロキシは根本的にネットワークにさらなる複雑さをもたらす.
例えば,SR プロキシは NF から返されるパケットに付加する適切な SID リストを決定する必要がある.
内部パケットは任意の宛先と送信元を持つ可能性があり,そのため SR プロキシが付けるべき SID リストは内部パケットによって異なる可能性がある.
SRプロキシは,適切なSIDリストを決定するための独自のメカニズムを実装する必要がある.
例えば,静的な SID リストをアタッチする End.AS か,プロキシの内部で状態をキャッシュする必要がある~\cite{sfc-proxy-bpf} .
さらに,SR プロキシをデプロイするためにはいくつかの問題が存在する.
例えば特定の SR プロキシタイプと共存できないサービスのタイプ,サービスの有効性の検出,SR プロキシの背後のサービスに対する SID 広告の問題など~\cite{draft-scexp}が既にインターネットドラフトとして挙げらている.


\textbf{[この段落では End.eBPF について述べる]}
\begin{itemize}
    \item eBFP とはなにか
    \begin{itemize}
        \item eBPF を使って NF を作ることができる 
        \item 最近では LKM に変わって Linux kernel を拡張する方法としての側面に注目が集まっている
        \item eBPF は Virtual Machine で動作するものの Linux kernel に依存する機能も多い
    \end{itemize}
    \item End.eBPF とはなにか
    \item eBPF に対して,本研究では netfilter を NF に統合することを目的としている
    \begin{itemize}
        \item ここで Linux kernel に依存する End ビヘイビアを提案することの妥当性を述べる
    \end{itemize}
\end{itemize}

\section{問題提起}
\label{section:prob}
現在,Linux の持つ SRv6・IPv6 ルーティングインフラストラクチャを活用しながら Linux カーネルに実装されている netfilter という多機能なパケット処理機能 を NF として利用する手法は,確立されていない.
現在提案されている手法では,Linux の持つ IPv6 ルーティングインフラストラクチャを活用した SR-Aware NF をシンプルに実現することは難しい.
また,Linux カーネルには netfilter という多機能なパケット処理機能が実装されているものの,SRv6 上で netfilter を直接 NF として扱う方法も確立されていない.
SRv6 は NF を SID として表すことで SFC を実現可能なアーキテクチャであるものの,SID として表現された SRv6 上のノードとしての NF と,実際のアプリケーションとしての NF を統合するの方法は自明ではない.
セクション~\ref*{section:srv6}で述べた SR プロキシを利用する方法では,SR プロキシを導入することで生まれるオーバヘッドや運用上の問題が指摘されている.
また,SRH でカプセル化されているパケットに対して,パケットをカプセル化したまま netfilter を適用する手法も確立されていない.
本論文では,Linux netfilter を SRv6 NF として活用するための手法を提案する.

\section{Linux netfilter}
\textbf{[Linux netfilter について説明する. netfilter によってできることや hook point の場所やパケットの流れについて説明する.]}
\textbf{[なぜ本研究では NF として netfilter を選んだのかについても述べる.]}